\documentclass{article}

% Language setting
% Replace `English' with e.g. `Spanish' to change the document language
\usepackage[english]{babel}
\usepackage{ctex}

% Set page size and margins
% Replace `letter paper' with`a4paper' for UK/EU standard size
\usepackage[a4paper,top=1.5cm,bottom=1.5cm,left=2cm,right=2cm,marginparwidth=1.75cm]{geometry}

% Useful packages
\usepackage{amsmath}
\usepackage{graphicx}
\usepackage{listings}
\usepackage[ruled]{algorithm2e}
\usepackage[colorlinks=true, allcolors=blue]{hyperref}

\SetKwProg{Function}{function}{:}{end}

% TODO 个人信息参数全局变量
% -------------- define personal variables begin --------------
\newcommand\myCollege{[college name]}
\newcommand\myId{[author id]}
\newcommand\myName{[author name]}
\newcommand\myHomeworkId{[homework id]}
% --------------  define personal variables end  --------------

\title{\heiti \myCollege《算法设计与分析》第\myHomeworkId 次作业} % 文章标题
\author{\myCollege \quad \myId \quad \myName} % 作者信息

\begin{document}
\maketitle

\section{[作业题目1]} % TODO 作业题目1名字

\subsection{[作业题目1.1]} % TODO 作业题目1.1名字

% TODO 大括号样例

$$
T(n)=\left\{
\begin{array}{l}
1\ ,\ n=1\\
T(n-2)+3n\ ,\ n>1
\end{array}
\right.
$$

$T(n)=T(n-2)+3n=T(n-4)+3n+3(n-2)=T(n-6)+3n+3(n-2)+3(n-4)=...$

\qquad\ $=3n+3(n-2)+3(n-4)+...+9+1=\frac{1}{2}(3n+9)(\frac{3n-9}{6}+1)+1=O(n^2)$

由此可知,原式渐进上界为$O(n^2)$。



\qquad$n=1$时,对$c\ge1$结论显然成立;

\qquad$n>1$时,有:$T(n)=T(n/2)+2^n$

\qquad\qquad\qquad\qquad\qquad\quad$\le{c}2^\frac{n}{2}+2^n$

\qquad\qquad\qquad\qquad\qquad\quad$=c2^n-(c-1)2^n+c2^\frac{n}{2}$

\qquad\qquad\qquad\qquad\qquad\quad$=c2^n-((c-1)2^\frac{n}{2}-c)2^\frac{n}{2}$

\qquad\qquad\qquad\qquad\qquad\quad$\le{c}2^n-(2(c-1)-c)2^\frac{n}{2}$

\qquad\qquad\qquad\qquad\qquad\quad$=c2^n-(c-2)2^\frac{n}{2}$

显然,只要$c\ge2$,就有$T(n)\le{c2^n}$成立,从而原式渐进上界为$O(2^n)$。

\subsection{[作业题目1.2]} % TODO 作业题目1.2名字

$$
T(n)=\left\{
\begin{array}{l}
1\ ,\ n=1\\
8T(n/4)+2n\ ,\ n>1
\end{array}
\right.
$$

使用主方法计算渐进上界,注意到$2n=O(n)$,由上式可知$a=8,b=4,d=1$,由$d<log_b a$,可得上式的渐进上界为$O(n^{log_b a})=O(n^\frac{3}{2})$。


\section{[作业题目2]} % TODO 作业题目2名字

% TODO 作业题目2描述

现有$k$个有序数组(从小到大排序),每个数组中包含$n$个元素。你的任务是将他们合并成$1$个包含$kn$个元素的有序数组。首先来回忆一下课上讲的归并排序算法,它提供了一种合并有序数组的算法$Merge$。如果我们有两个有序数组的大小分别为$x$和$y$,$Merge$算法可以用$O(x+y)$的时间来合并这两个数组。

\subsection{[作业题目2分析]} % TODO 作业题目2分析

如果我们应用$Merge$算法先合并第一个和第二个数组,然后由合并后的数组与第三个合并,再与第四个合并,直到合并完$k$个数组。请分析这种合并策略的时间复杂度(请用关于$k$和$n$的函数表示)。

显然,第一次合并的复杂度为$O(n+n)$,第二次合并的复杂度为$O(2n+n)$,第三次合并的复杂度为$O(3n+n)$,以此类推,第$k-1$次合并的复杂度为$O((k-1)n+n)$。

从而总复杂度为$n+2n+...+(k-1)n=\frac{1}{2}k(k-1)n=O(nk^2)$。

\subsection{[作业题目2算法与伪代码]} % 作业题目2算法与伪代码

针对本题的任务,请给出一个更高效的算法,并分析它的时间复杂度(此题若取得满分,所设计算法的时间复杂度应为$O(n k log k)$)。

分析题目,只要使用优先队列维护$k$个有序数组未归并部分首个元素的集合即可,每次从优先队列中取出一个元素归并,再将提供被取出元素的数组的下一元素插入优先队列,这样共需要处理$k n$个元素,又因为优先队列的大小为$k$,每个元素的处理时间为$log k$,即得到$O(n k log k)$的时间复杂度,伪代码如下。

% TODO 伪代码模板

\begin{algorithm}[H]

\caption{k路归并问题优化做法}
\LinesNumbered
\KwIn{正整数$n$,为数组长度,正整数$k$,为数组个数,$k$个长度为$n$的数组$A[]$}
\KwOut{长度为$n k$的数组$Ans$,为归并后的数组}

$global\ pointer[k]$

\Function{$main(n,k,A)$}{
    \For{$i\leftarrow0\ to\ k-1$}{
        $pointer[i] \leftarrow 0$\;
        $push\ (A[i][pointer[i]],i)\ to\ priority\_queue$\;
    }
    \For{$i\leftarrow0\ to\ n k-1$}{
        $(val,index) \leftarrow top\ of\ priority\_queue$\;
        $remove\ top\ from\ priority\_queue$\;
        $Ans[i] \leftarrow val$\;
        $pointer[index] \leftarrow pointer[index]+1$\;
        \If{$pointer[index]<{length\ of\ A[index]}$}{
            $push\ (A[index][pointer[index]],index)\ to\ priority\_queue$\;
        }
    }
    \textbf{$return$}\; 
}

\end{algorithm}

\begin{algorithm}[H]

\caption{填数字问题朴素做法}
\LinesNumbered
\KwIn{正整数$n$,为数组长度}
\KwOut{长度为$n$的数组$A$,为填充后的数组}

\Function{$main(n)$}{
    $push\ (0,n-1)\ to\ priority\_queue$\;
    \For{$i\leftarrow1\ to\ n$}{
        $(l,r) \leftarrow top\ of\ priority\_queue$\;
        $remove\ top\ from\ priority\_queue$\;
        $mid \leftarrow (l+r)/2$\;
        $A[mid] \leftarrow i$\;
        \If{$mid-1\ge{l}$}{
            $push\ (l,mid-1)\ to\ priority\_queue$\;
        }
        \If{$r\ge{mid+1}$}{
            $push\ (mid+1,r)\ to\ priority\_queue$\;
        }
    }
    \textbf{$return\ A$}\; 
}

\end{algorithm}

\begin{algorithm}[H]

\caption{填数字问题优化做法}
\LinesNumbered
\KwIn{正整数$n$,为数组长度}
\KwOut{长度为$n$的数组$A$,为填充后的数组}

$global\ list[n]$\;

\Function{$search(l,r)$}{
    $mid \leftarrow (l+r)/2$\;
    $insert\ mid\ to\ head\ of\ list[r-l+1]$\;
    \If{$r\ge{mid+1}$}{
        $call\ search(mid+1,r)$\;
    }
    \If{$mid-1\ge{l}$}{
        $call\ search(l,mid-1)$\;
    }
}

\Function{$main(n)$}{
    $call\ search(0,n-1)$\;
    $count \leftarrow 1$\;
    \For{$i\leftarrow{n}\ to\ 1$}{
        \For{$j\leftarrow{head\ of\ list[i]}\ to\ tail\ of\ list[i]$}{
            $A[value\ of\ j] \leftarrow count$\;
            $count \leftarrow count+1$\;
        }
    }
    \textbf{$return\ A$}\; 
}

\end{algorithm}

\section{[附录暨部分算法的实现]} %TODO 代码附录

\subsection{填数字问题朴素做法与测试数据生成}

\definecolor{mygreen}{rgb}{0,0.6,0}
\definecolor{mygray}{rgb}{0.5,0.5,0.5}
\definecolor{mymauve}{rgb}{0.58,0,0.82}

\lstset{
    backgroundcolor=\color{white},   % choose the background color; you must add \usepackage{color} or \usepackage{xcolor}
    basicstyle=\footnotesize,        % the size of the fonts that are used for the code
    breakatwhitespace=false,         % sets if automatic breaks should only happen at whitespace
    breaklines=true,                 % sets automatic line breaking
    captionpos=bl,                   % sets the caption-position to bottom
    commentstyle=\color{mygreen},    % comment style
    deletekeywords={...},            % if you want to delete keywords from the given language
    escapeinside={\%*}{*)},          % if you want to add LaTeX within your code
    extendedchars=true,              % lets you use non-ASCII characters; for 8-bits encodings only, does not work with UTF-8
    frame=single,                    % adds a frame around the code
    keepspaces=true,                 % keeps spaces in text, useful for keeping indentation of code (possibly needs columns=flexible)
    keywordstyle=\color{blue},       % keyword style
    language=c++,                    % the language of the code
    morekeywords={*,...},            % if you want to add more keywords to the set
    numbers=left,                    % where to put the line-numbers; possible values are (none, left, right)
    numbersep=5pt,                   % how far the line-numbers are from the code
    numberstyle=\tiny\color{mygray}, % the style that is used for the line-numbers
    rulecolor=\color{black},         % if not set, the frame-color may be changed on line-breaks within not-black text (e.g. comments (green here))
    showspaces=false,                % show spaces everywhere adding particular underscores; it overrides 'showstringspaces'
    showstringspaces=false,          % underline spaces within strings only
    showtabs=false,                  % show tabs within strings adding particular underscores
    stepnumber=1,                    % the step between two line-numbers. If it's 1, each line will be numbered
    stringstyle=\color{mymauve},     % string literal style
    tabsize=2,                       % sets default tabsize to 2 spaces
}

\begin{lstlisting}
#include <queue>
#include <cstdio>
using namespace std;

struct Node
{
	int l,r,len;

	inline Node(int l,int r)
	{
		this->l=l,this->r=r;
		len=r-l+1; return;
	}

	inline bool operator<(const Node& rhs) const
	{
		if (this->len<rhs.len) return true;
		if (this->len>rhs.len) return false;
		return this->l>rhs.l;
	}

	inline bool operator>(const Node& rhs) const
	{
		if (this->len>rhs.len) return true;
		if (this->len<rhs.len) return false;
		return this->l<rhs.l;
	}
};

int a[10005];

int main()
{
	int n=10000;
	freopen("testcase.txt","w",stdout);
	printf("%d\n",n);
	priority_queue<Node> pq;
	Node node(0,n-1); pq.push(node);
	for (int i=1,l,r;i<=n;i++)
	{
		Node node=pq.top(); pq.pop();
		l=node.l,r=node.r;
		// printf("%d %d\n",l,r);
		int mid=(l+r)>>1;
		a[mid]=i;
		if (mid-1>=l) {Node newNode(l,mid-1); pq.push(newNode);}
		if (r>=mid+1) {Node newNode(mid+1,r); pq.push(newNode);}
	}
	for (int i=0;i<n;i++)
	printf("%d\n",a[i]);
	return 0;
}
\end{lstlisting}

\subsection{填数字问题优化做法与自动对拍}

\begin{lstlisting}
#include <cstdio>
#include <vector>
using namespace std;

int a[10005];
int tot,fst[10005],nxt[10005],val[10005];

void build(int l,int r)
{
	nxt[++tot]=fst[r-l+1],fst[r-l+1]=tot;
	int mid=(l+r)>>1; val[tot]=mid;
	if (r>=mid+1) build(mid+1,r);
	if (mid-1>=l) build(l,mid-1);
	return;
}

int main()
{
	int n,cnt=0;
	freopen("testcase.txt","r",stdin);
	scanf("%d",&n);
	build(0,n-1);
	for (int i=n;i>0;i--)
	for (int j=fst[i];j;j=nxt[j])
	a[val[j]]=++cnt;
	for (int i=0;i<n;i++) printf("%d\n",a[i]);
	bool isRight=true;
	for (int i=0,std;i<n;i++)
	{
		scanf("%d",&std);
		if (a[i]!=std) isRight=false;
	}
	printf(isRight?"AC\n":"WA\n");
	return 0;
}
\end{lstlisting}

\end{document}